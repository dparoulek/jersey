%
% DO NOT ALTER OR REMOVE COPYRIGHT NOTICES OR THIS HEADER.
% 
% Copyright 1997-2010 Sun Microsystems, Inc. All rights reserved.
% 
% The contents of this file are subject to the terms of either the GNU
% General Public License Version 2 only ("GPL") or the Common Development
% and Distribution License("CDDL") (collectively, the "License").  You
% may not use this file except in compliance with the License. You can obtain
% a copy of the License at https://jersey.dev.java.net/CDDL+GPL.html
% or jersey/legal/LICENSE.txt.  See the License for the specific
% language governing permissions and limitations under the License.
% 
% When distributing the software, include this License Header Notice in each
% file and include the License file at jersey/legal/LICENSE.txt.
% Sun designates this particular file as subject to the "Classpath" exception
% as provided by Sun in the GPL Version 2 section of the License file that
% accompanied this code.  If applicable, add the following below the License
% Header, with the fields enclosed by brackets [] replaced by your own
% identifying information: "Portions Copyrighted [year]
% [name of copyright owner]"
% 
% Contributor(s):
% 
% If you wish your version of this file to be governed by only the CDDL or
% only the GPL Version 2, indicate your decision by adding "[Contributor]
% elects to include this software in this distribution under the [CDDL or GPL
% Version 2] license."  If you don't indicate a single choice of license, a
% recipient has the option to distribute your version of this file under
% either the CDDL, the GPL Version 2 or to extend the choice of license to
% its licensees as provided above.  However, if you add GPL Version 2 code
% and therefore, elected the GPL Version 2 license, then the option applies
% only if the new code is made subject to such option by the copyright
% holder.

\documentclass{acm_proc_article-sp}

\begin{document}
\toappear{Permission to make digital or hard copies of all or part of this work for personal or classroom use is granted without fee provided that copies are not made or distributed for profit or commercial advantage and that copies bear this notice and the full citation on the first page. To copy otherwise, to republish, to post on servers or to redistribute to lists, requires prior specific permission and/or a fee.

WS-REST 2010, April 26 2010, Raleigh, NC, USA

Copyright � 2010 ACM 978-1-60558-959-6/10/04... \$10.00}

\title{Exploring Hypermedia Support in Jersey}

\numberofauthors{3}
\author{
\alignauthor Marc Hadley\\
       \affaddr{Oracle}\\
       \affaddr{Burlington, MA, USA}\\
       \email{\normalsize marc.hadley@sun.com}
\alignauthor Santiago Pericas-Geertsen\\
       \affaddr{Oracle}\\
       \affaddr{Palm Beach Grdns, FL, USA}\\
       \email{\normalsize santiago.pericasgeertsen@sun.com}
\alignauthor Paul Sandoz\\
       \affaddr{Oracle}\\
       \affaddr{Grenoble, France}\\
       \email{\normalsize paul.sandoz@sun.com}
}

\date{6 January 2010}

\maketitle
\begin{abstract}
This paper describes a set of experimental extensions for Jersey\cite{jersey} that aim to simplify server-side creation and client-side consumption of hypermedia-driven services. We introduce the concept of {\em action resources} that expose workflow-related operations on a parent resource.
\end{abstract}

\category{D.3.3}{Programming Languages}{Language Constructs and Features}
\category{H.4.3}{Information Systems Applications}{Communication Applications}
\category{H.5.4}{Information Interfaces and Presentation}{Hypertext/Hypermedia}

\section{Introduction}
\label{introduction}
The REST architectural style, as defined by Roy Fielding in his thesis \cite{rest}, is 
characterized by 
four constraints: (i) identification of resources (ii) manipulation of resources through 
representations (iii) self-descriptive messages and (iv) hypermedia as the engine of 
application state. It is constraint (iv), hypermedia as the engine of application state or
HATEOAS for short, that is the least understood and the focus of this paper.
HATEOAS refers to the use of {\em hyperlinks} in resource representations as a way of
navigating the state machine of an application.

It is generally understood that, in order to follow the REST style, URIs
should be assigned to anything of interest (resources) and a few, well-defined
operations (e.g., HTTP operations) should be used to interact with
these resources. For example, the state of a purchase order resource
can be updated by POSTing (or PATCHing) a new value for its
{\em state} field.\footnote{The choice of operation, such as POST, PUT
or PATCH, for this type of update is still a matter of debate. See 
\cite{fielding:post}.} However, 
as has been identified by other authors \cite{bray}\cite{guilherme},
there are {\em actions} that cannot be easily mapped to read
or write operations on resources. These operations are inherently more 
complex and their details are rarely of interest to clients. For example,
given a {\em purchase order resource}, the operation of
setting its state to {\tt REVIEWED} may involve a number
of different steps such as (i) checking the customer's credit status (ii) 
reserving inventory and (iii) verifying per-customer quantity limits. Clearly,
this {\em workflow} cannot be equated to simply updating a field
on a resource. Moreover, clients are generally uninterested in the details
behind these type of workflows, and in some cases computing
the final state of a resource on the client side, as required for a
PUT operation, is impractical or impossible.\footnote{For instance,
when certain parts of the model needed by the workflow are not
exposed as resources on the client side.}

This paper introduces the concept of {\em action resources} and
explores extensions to Jersey \cite{jersey} to support them. An 
action resource is a sub-resource defined for the purpose of
exposing workflow-related operations on parent resources.  As
sub-resources, action resources are identified by URIs that are relative to their
parent. For instance, the following are examples of action resources:
\begin{verbatim}
http://.../orders/1/review
http://.../orders/1/pay
http://.../orders/1/ship
\end{verbatim}
for purchase order ``1" identified by {\tt http://.../orders/1}.

Action resources provide a simplified hypertext model that 
can be more easily supported by generic frameworks like Jersey.
A set of action resources defines\textemdash via their link
relationships\textemdash 
a {\em contract} with clients that has the potential to evolve
over time depending on the application's state. For instance,
assuming purchase orders are only reviewed once,
the {\tt review} action will become unavailable and the
{\tt pay} action will become available after an order is 
reviewed.

The notion of action resources naturally leads to discussions
about improved {\em client} APIs to support them. Given that
action resources are identified by URIs, no additional API
is really necessary, but the use of client-side  proxies and 
method invocations to trigger these actions seems quite
natural \cite{guilherme}. Additionally, the use of client proxies
introduces a level of indirection that enables better support
for {\em server evolution}, i.e.~the ability of a server's contract
to support certain changes without breaking existing clients. 
Finally, it has been argued \cite{craig} that using client proxies
is simply more natural for developers and less error prone as
fewer URIs need to be constructed.

%\newpage 
\section{Types of Contracts}
A contract established between a server and its clients 
can be {\em static} or {\em dynamic}. In a static contract, knowledge about the 
server's model is embedded into clients and cannot be updated
without re-writing them. In a dynamic contract, clients are capable of 
discovering knowledge about the contract at runtime and adjust accordingly.

In addition, a dynamic contract can be further subdivided into {\em contextual}
and {\em non-contextual}. Contextual contracts can be updated in the 
course of a conversation depending on the application's state; conversely, 
non-contextual contracts are fixed and independent of the application's state.

HATEOAS is characterized by the use of contextual contracts where
the set of actions varies over time. In our purchase ordering system
example, this contextual contract will prevent the {\tt ship} action
to be carried out before the order is paid, i.e.~before the {\tt pay}
action is completed.

\section{Human Vs. Bots}
\label{human-vs-bots}
In order to enable servers to evolve independently, clients and servers 
should be as decoupled as possible and everything that can change should be learned
on the fly. 
The {\em Human Web} is based on this type of highly dynamic contracts 
in which very little is known {\em a priori}. As very adaptable creatures, humans 
are able to quickly learn new contracts (e.g.~a new login page to access a
bank account) and maintain compatibility.

 In the {\em Bot Web}, on the 
other hand, contracts
are necessarily less dynamic and must incorporate some static knowledge
as part of the bot's programming: a bot that dynamically learns about some action
resources will not be able to {\em choose} which one to use next if that
decision is not part of its programming. It follows that at least a subset of the
server's state machine\textemdash of which actions are transitions\textemdash must be
statically known for the bot to accomplish some pre-defined task. However,
if action descriptions (including HTTP methods, URI, query parameters, etc.)  are mapped at 
runtime, then they need not be statically known. In fact, 
several degrees of coupling can be supported as part of the same
framework depending on how much information is available 
statically vs.~dynamically.

\section{Hypermedia in Jersey}
\label{hypermedia-in-jersey}
Jersey~\cite{jersey} is the reference implementation of the Java API for
RESTful Web Services~\cite{jaxrs11}. In this section, we shall describe
experimental extensions developed for Jersey to support HATEOAS, 
including a new client API based on Java dynamic proxies. 

These Jersey extensions were influenced by the following (inter-related)
requirements:
\begin{description}
\item[HATEOAS] Support for {\em actions} and {\em contextual
action sets} as first-class citizens. 
\item[Ease of use] Annotation-driven model for both client APIs and 
server APIs. Improved client API based on dynamic generation
of Java proxies.
\item[Server Evolution] Various degrees of client and server
coupling, ranging from static contracts to 
contextual contracts. 
\end{description}

Rather than presenting all these extensions abstractly, we shall
illustrate their use via an example. The {\em Purchase 
Ordering System} exemplifies a system in which customers
can submit orders and where orders are guided by a workflow that
includes states like {\tt REVIEWED}, {\tt PAID} and {\tt SHIPPED}.

%% Insert table describing an order, states and transitions

The system's model is comprised of 4 entities: {\tt Order}, 
{\tt Product}, {\tt Customer} and {\tt Address}. These model entities are 
controlled by 3 resource classes: {\tt OrderResource}, 
{\tt CustomerResource} and {\tt ProductResource}. Addresses are 
sub-resources that are also controlled by {\tt CustomerResource}.
An order instance refers to a single customer, a single 
address (of that customer) and one or more products. The XML
representation (or view) of a sample order is shown 
below.\footnote{Additional whitespace was added for clarity and
space restrictions.}

{\small
\begin{verbatim}
<order>
    <id>1</id>
    <customer>http://.../customers/21</customer>
    <shippingAddress>
        http://.../customers/21/address/1
    </shippingAddress>
    <orderItems>
        <product>http://.../products/3345</product>
        <quantity>1</quantity>
    </orderItems>
    <status>RECEIVED</status>
</order>
\end{verbatim}}

Note the use of URIs to refer to each component of an order. This
form of {\em serialization by reference} is supported in Jersey
using JAXB beans and the {\tt @XmlJavaTypeAdapter} 
annotation.\footnote{This annotation can be used
to customize marshalling and unmarshalling using {\tt XmlAdapter}'s.
An {\tt XmlAdapter} is capable of mapping an object reference
in the model to a URI.}

\subsection{Server API}
The server API introduces 3 new annotation types:  {\tt @Action}, 
  {\tt @ContextualActionSet} and {\tt @HypermediaController}. 
The {\tt @Action} annotation identifies a sub-resource as a
{\em named action}. The {\tt @ContextualActionSet}
is used to support contextual contracts and must annotate
a method that returns a set of action names. Finally, 
 {\tt @HypermediaController}
marks a resource class as a {\em hypermedia controller
class}: a class with one more methods annotated
with {\tt @Action} and at most one method annotated
with {\tt @ContextualActionSet}.

The following example illustrates the use of all these
annotation types to define the {\tt OrderResource}
controller.\footnote{Several details about this class
are omitted for clarity and space restrictions.}
{\small
\begin{verbatim}
@Path("/orders/{id}")
@HypermediaController(
    model=Order.class,
    linkType=LinkType.LINK_HEADERS)
public class OrderResource {

    private Order order;

    @GET @Produces("application/xml")
    public Order getOrder(
      @PathParam("id") String id) {      
        return order;
    }

    @POST @Action("review") @Path("review")
    public void review(
      @HeaderParam("notes") String notes) {
        ...
        order.setStatus(REVIEWED);
    }

    @POST @Action("pay") @Path("pay")
    public void pay(
      @QueryParam("newCardNumber") String newCardNumber) {
        ...
        order.setStatus(PAID);
    }

    @PUT @Action("ship") @Path("ship")
    @Produces("application/xml")
    @Consumes("application/xml")
    public Order ship(Address newShippingAddress) {
        ...
        order.setStatus(SHIPPED);
        return order;
    }

    @POST @Action("cancel") @Path("cancel")
    public void cancel(
      @QueryParam("notes") String notes) {
        ...
        order.setStatus(CANCELED);
    }
}
\end{verbatim}}

The {\tt @HypermediaController} annotation above indicates that this
resource class is a hypermedia controller for the {\tt Order} class. 
Each method annotated with {\tt @Action} defines a link
relationship and associated action resource. These methods
are also annotated with {\tt @Path} to make a 
sub-resource.\footnote{There
does not appear to be a need to use {\tt @Action} and
{\tt @Path} simultaneously, but without the latter some
resource methods may become ambiguous. In the future,
we hope to eliminate the use of {\tt @Path} when {\tt @Action}
is present.}
In addition, {\tt linkType} selects the way in which URIs corresponding
to action resources are serialized: in this case, using link 
headers \cite{linkheaders}. These link headers become part of
the order's representation. For instance, an order in the
{\tt RECEIVED} state, i.e.~an order that can only be reviewed
or canceled, will be represented as follows.\footnote{Excluding
all other HTTP headers for clarity.}

{\small
\begin{verbatim}
Link: <http://.../orders/1/review>;rel=review;op=POST
Link: <http://.../orders/1/cancel>;rel=cancel;op=POST
<order>
    <id>1</id>
    <customer>http://.../customers/21</customer>
    <shippingAddress>
        http://.../customers/21/address/1
    </shippingAddress>
    <orderItems>
        <product>http://.../products/3345</product>
        <quantity>1</quantity>
    </orderItems>
    <status>RECEIVED</status>
</order>
\end{verbatim}}

Link headers, rather than links embedded within the entity, were chosen for expediency. A full-featured framework would support both link headers and links embedded within entities but, for the purposes of this investigation, having links only in headers allowed for simpler, media-type independent, link extraction machinery on the client-side.

Without a method annotated with {\tt @ContextualActionSet},
all actions are available at all times regardless
of the state of an order. The following method can be
provided to define a contextual contract for this resource.

{\small
\begin{verbatim}
@ContextualActionSet
public Set<String> getContextualActionSet() {
    Set<String> result = new HashSet<String>();
    switch (order.getStatus()) {
        case RECEIVED:
            result.add("review");
            result.add("cancel");
            break;
        case REVIEWED:
            result.add("pay");
            result.add("cancel");
            break;
        case PAID:
            result.add("ship");
            break;
        case CANCELED:
        case SHIPPED:
            break;
    }
    return result;
}
\end{verbatim}}

This method returns a set of action names based on the
order's internal state; the values returned in this set
correspond to the {\tt @Action} annotations in the
controller. For example, this contextual contract prevents
shipping an order that has not been paid by only
including the {\tt ship} action in the {\tt PAID} state.

Alternate declarative approaches for contextualizing action sets
were also investigated but the above approach was chosen for 
this investigation due to its relative simplicity and expediency.

\subsection{Client API}
Although action resources can be accessed using
traditional APIs for REST, including Jersey's client API \cite{jersey},
the use of client-side proxies and method invocations to
trigger these actions seems quite natural. As we shall see,
the use of client proxies also introduces a level of indirection that 
enables better support for server evolution, permitting the
definition of contracts with various degrees of coupling.

Client proxies are created based on {\em hypermedia
controller interfaces}. Hypermedia controller interfaces
are Java interfaces annotated by {\tt @HypermediaController}
that, akin to the server side API, specify the name of a
model class and the type of serialization to use for
action resource URIs. The client-side model class
should be based on the representation returned by
the server; in particular, in our example the client-side
model for an {\tt Order} uses instances of {\tt URI} to link
an order to a customer, an address and a list of
products. 

{\small
\begin{verbatim}
@HypermediaController(
    model=Order.class,
    linkType=LinkType.LINK_HEADERS)
public interface OrderController {

    public Order getModel();

    @Action("review")
    public void review(@Name("notes") String notes);

    @POST @Action("pay")
    public void pay(@QueryParam("newCardNumber") 
        String newCardNumber);

    @Action("ship")
    public Order ship(Address newShippingAddress);

    @Action("cancel")
    public void cancel(@Name("notes") String notes);

}
\end{verbatim}}

The {\tt @Action} annotation associates an interface method
with a link relation and hence an action resource on the server.
Thus, invoking a
method on the generated proxy results
in an interaction with the corresponding action resource. The way in
which the method invocation is mapped to an HTTP
request depends on the additional annotations specified
in the interface. For instance, the {\tt pay} action in the
example above indicates that it must use a {\tt POST} 
and that the {\tt String} parameter {\tt newCardNumber} must
be passed as a query parameter. This is an example of a
{\em static} contract in which the client has built-in knowledge
of the way in which an action is defined by the server.
RESTeasy \cite{resteasy} provides a client API that follows this
model. 

In contrast, the {\tt review} action only provides
a name for its parameter using the {\tt @Name} annotation.
This is an example of a {\em dynamic} contract in which
the client is only coupled to the {\tt review} link relation
and the knowledge that this relation requires {\tt notes} to be supplied.
The exact interaction with the {\tt review}
action must therefore be discovered dynamically and
the {\tt notes} parameter mapped accordingly. The 
Jersey client runtime uses WADL fragments,
dynamically generated by the server, that describe action
resources to map these method calls into HTTP requests. 
The following shows the WADL description of the {\tt review} 
action resource:

{\small
\begin{verbatim}
<?xml version="1.0" encoding="UTF-8" standalone="yes"?>
<application xmlns="http://wadl.dev.java.net/2009/02"
    xmlns:xs="http://www.w3.org/2001/XMLSchema">
  <doc xmlns:jersey="http://jersey.dev.java.net/" 
    jersey:generatedBy="Jersey: ..."/>
  <resources base="http://localhost:9998/">
    <resource path="orders/1/review">
      <method name="POST" id="review">
        <request>
          <param type="xs:string" style="header" 
            name="notes"/>
        </request>
      </method>
    </resource>
  </resources>
</application>\end{verbatim}}

Essentially the WADL is used as a form to configure the 
request made by the client runtime. The WADL defines the 
appropriate HTTP method to use ({\tt POST} in this case) and 
the value of the {\tt @Name} annotation is matched to the 
corresponding parameter in the WADL to identify where in
the request to serialize the value of the {\tt notes}
parameter. 

The following sample shows how to use
{\tt OrderController} to generate a proxy
to review, pay and ship an order. For the sake
of the example, we assume the customer that
submitted the order has been suspended and
needs to be activated before the order is reviewed.
For that purpose, the client code retrieves the
customer URI from the order's model and 
obtains an instance of 
{\tt CustomerController}.\footnote{{\tt CustomerController}
is a hypermedia controller interface akin to {\tt OrderController}
which is omitted since it does not highlight any additional 
feature.}


{\small
\begin{verbatim}
// Instantiate Jersey's Client class
Client client = new Client();

// Create proxy for order and retrieve model
OrderController orderCtrl = client.proxy(
    "http://.../orders/1", OrderController.class);

// Create proxy for customer in order 1
CustomerController customerCtrl = client.proxy(
    orderCtrl.getModel().getCustomer(),
    CustomerController.class);

// Activate customer in order 1 
customerCtrl.activate();

// Review and pay order 
orderCtrl.review("approve");
orderCtrl.pay("123456789");

// Ship order 
Address newAddress = getNewAddress();
orderCtrl.ship(newAddress);
\end{verbatim}}

The client runtime will automatically
update the action set throughout a conversation: 
for example, even though
the {\tt review} action does not produce
a result, the HTTP response to that action 
still includes a list of link headers defining
the contextual action set, which in this
case will consist of the {\tt pay} and {\tt cancel}
actions but not the {\tt ship} action. An 
attempt to interact with any action not in the
context will result in a client-side exception.

\subsection{Server Evolution}
Section \ref{hypermedia-in-jersey} listed server evolution
and the ability to support degrees of coupling as a requirement
for our solution. In the last section, we have seen how the use
of client proxies based on {\em partially} annotated interfaces
facilitates server evolution. 

An interface method annotated with
{\tt @Action} and {\tt @Name} represents a {\em loose} contract with
a server.  Changes to action resource URIs, HTTP methods and 
parameter types on the server will not require a client re-spin. 
Naturally, as in all client-server architectures, it is always possible
to break backward compatibility, but the ability to support 
more dynamic contracts\textemdash usually at the cost of additional
processing time\textemdash is still an area of investigation. We see
our contribution as a small step in this direction, showing the
potential of using dynamic meta-data (WADL in our case) for the
definition of these type of contracts.

\section{Related Work}
In this section we provide a short overview of other REST frameworks
that inspired our work. RESTeasy \cite{resteasy} provides a framework that supports
client proxies generated from annotated interfaces. RESTfulie \cite{restfulie} is, to
the best of our knowledge, the first public framework with
built-in support for hypermedia.

\subsection{RESTeasy Client Framework}
The RESTeasy Client Framework follows a similar approach in the 
use of client-side annotations on Java interfaces for the creation of dynamic 
proxies. It differs from our approach in that it neither supports hypermedia
nor the ability to map proxy method calls using dynamic information. That is,
in the RESTeasy Client Framework,
proxy method calls are mapped to HTTP requests exclusively using 
static annotations. These client-side annotations establish a tight coupling
that makes server evolution difficult.

Despite these shortcomings, we believe their approach is a good match
for certain types of applications, especially those in which servers and
clients are controlled by the same organization. In addition, the use of client
proxies simplifies programming and improves developer
productivity. For these reasons, we have followed a similar programming
model
while at the same time provided support for hypermedia and
dynamic contracts.

\subsection{RESTfulie Hypermedia Support}
This article \cite{guilherme} by Guillerme S. describes how to implement
hypermedia aware resources using RESTfulie. In RESTfulie, action URIs 
are made part of a resource representation (more specifically, the
entity) via the use of Atom links \cite{atom}.  Even though we
foresee supporting other forms of URI serialization (as
indicated by the use of {\tt linkType} in {\tt @HypermediaController}),
we believe link headers to be the least intrusive and most
likely to be adopted when migrating existing applications into
hypermedia-aware ones.\footnote{For example, existing applications
that use XML schema validation on the entities.} 

Rather than providing an explicit binding between actions and 
HTTP methods, RESTfulie provides a pre-defined table that maps
{\tt rel} elements to HTTP methods. For example, an {\tt update}
action is mapped to {\tt PUT} and a {\tt destroy} action is mapped
to {\tt DELETE}. We believe this implicit mapping to be unnecessarily
confusing and not easily extensible. Instead, as we have done
in this paper, we prefer to make {\em action resources} first
class and provide developers tools to define explicit mappings
via the use of {\tt @Action} annotations.

In RESTfulie, knowledge about action resources is discovered 
dynamically and, as a result, 
Java reflection is the only mechanism available to interact
with hypermedia-aware resources.\footnote{The exact
manner in which the RESTfulie runtime maps parameters 
in an HTTP request is unclear to us at the time of writing.}
So instead of writing
{\tt order.cancel()}, a developer needs to write:

\begin{verbatim} 
  resource(order).getTransition("pay").execute()
\end{verbatim}

 As explained in
Section~\ref{human-vs-bots}, this decision is 
impractical given that certain static knowledge is
required in order to program bots. We believe that
Java interfaces annotated with {\tt @Action} and {\tt @Name}
provide the right amount of static
information to enable bot programming and server
evolution whilst not sacrificing the ease of use
offered by client proxies.

\section{Conclusions}
In this paper we have introduced the notion of an 
action resource and, with it, described some experimental
extensions for Jersey to support HATEOAS.
Our analysis
lead us into the exploration of innovative client APIs 
that enable developers to define client-server contracts
with different degrees of coupling and improve the
ability of servers to evolve independently. In the process, 
we also argued that there
is a minimum amount of static information that is needed
to enable programmable clients (bots) and showed how
that information can be captured using client interfaces. 

All the source code shown in the previous sections is 
part of a complete and runnable sample \cite{sample}
available in the Jersey subversion repository.
The solution proposed herein is experimental and is
likely to evolve in unforeseen directions once developers
start exploring HATEOAS in real-world systems.

We are currently evaluating support for entity-based
action resources URIs. Support for Atom link elements
(as has been proposed by other authors) is a likely
avenue for exploration since their use, like that of
Link headers, permits support of many different
XML-based media types without requiring media-type
specific machinery. Different ways in which contextual
action sets are defined are being explored, as well as
simplifications to the client APIs for the instantiation
of dynamic proxies, especially those created from other proxies.

The authors would like to thank Martin Matula for 
his suggestions on how to improve
the sample and Gerard Davison for reviewing
our work and providing insightful comments.

%Generated by bibtex from your ~.bib file.  Run latex,
%then bibtex, then latex twice (to resolve references)
%to create the ~.bbl file.  Insert that ~.bbl file into
%the .tex source file.
%\bibliographystyle{abbrv}
%\bibliography{references}  % references.bib is the name of the Bibliography
\begin{thebibliography}{1}

\bibitem{burke:restfuljava}
B.~Burke.
\newblock {\em {RESTful} {Java} with {JAX-RS}}.
\newblock O'Reilly, 2009.

\bibitem{javaee6}
R.~Chinnici and B.~Shannon.
\newblock {Java} {Platform}, {Enterprise} {Edition} {(JavaEE)} {Specification},
  {v6}.
\newblock {JSR}, JCP, November 2009.
\newblock See http://jcp.org/en/jsr/detail?id=316.

\bibitem{rest}
R.~Fielding.
\newblock {Architectural} {Styles} and the {Design} of {Network-based}
  {Software} {Architectures}.
\newblock Ph.d dissertation, University of California, Irvine, 2000.
\newblock See http://roy.gbiv.com/pubs/dissertation/top.htm.

\bibitem{jaxrs11}
M.~Hadley and P.~Sandoz.
\newblock {JAXRS:} {Java} {API} for {RESTful} {Web} {Services}.
\newblock {JSR}, JCP, September 2009.
\newblock See http://jcp.org/en/jsr/detail?id=311.

\bibitem{bray}
T.~Bray.
\newblock RESTful Casuistry.
\newblock Blog, March 2009.
\newblock See http://www.tbray.org/ongoing/When/200x/2009/03/20/Rest-Casuistry.

\bibitem{fielding:post}
R.~Fielding.
\newblock It is okay to use POST.
\newblock Blog, March 2009.
\newblock See http://roy.gbiv.com/untangled/2009/it-is-okay-to-use-post.

\bibitem{guilherme}
G.~Silveira.
\newblock Quit pretending, use the web for real: restfulie.
\newblock Blog, November 2009.
\newblock See http://guilhermesilveira.wordpress.com/2009/11/03/quit-pretending-use-the-web-for-real-restfulie.

\bibitem{restfulie}
\newblock RESTfulie.
\newblock See http://freshmeat.net/projects/restfulie.

\bibitem{jersey}
\newblock JAX-RS reference implementation for building RESTful web services.
\newblock See https://jersey.dev.java.net.

\bibitem{craig}
\newblock Why HATEOAS?
\newblock Blog, April 2009.
\newblock See http://blogs.sun.com/craigmcc/entry/why\_hateoas.

\bibitem{linkheaders}
\newblock Web Linking (draft).
\newblock Mark Nottingham.
\newblock http://tools.ietf.org/html/draft-nottingham-http-link-header-06.

\bibitem{resteasy}
\newblock RESTeasy Client Framework.
\newblock See http://www.jboss.org/file-access/default/members/resteasy/freezone/docs/1.2.GA/\\userguide/html\_single/index.html\#RESTEasy\_Client\_Framework.

\bibitem{wadl}
\newblock Web Application Description Language (WADL).
\newblock Marc Hadley.
\newblock See https://wadl.dev.java.net.

\bibitem{atom}
\newblock Atom Syndication Format.
\newblock See http://www.w3.org/2005/Atom.

\bibitem{sample}
\newblock Jersey Hypermedia Sample.
\newblock See http://tinyurl.com/jersey-hypermedia.

\end{thebibliography}

\end{document}
